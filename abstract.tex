\begin{abstract}
Compilers and 
performance engineers use hardware performance models
to simplify program optimizations.
Performance models provide the necessary abstraction
over
the increasingly complex modern processors.
However, constructing and maintaining a performance model
is no easy feat, given the numerous microarchitectural
optimizations employed by modern processors.
Despite the complexity they face
and reported cases of their mispredictions (e.g. more then 30\% relative error),
there is no systematic validation of existing
performance models---such as IACA and llvm-mca---because there is no scalable machine code profiler
that can \textit{automatically} obtain throughput of arbitrary basic block while still conforming to common modelling assumptions.

In this paper, we present a novel profiler
that can profile \textit{arbitrary}, memory-accessing
basic blocks without any user-intervention. 
We used this profiler to build a benchmark for
the systematic validation of existing performance models
of x86-64 basic blocks.
Using this benchmark, 
we evaluated four existing performance models: IACA,
llvm-mca, Ithemal, and OSACA.
We describe how to automatically classify these basic blocks
based on their use of CPU resources.
Using this classification, our benchmark can give a detailed analysis of a performance model's strengths and weaknesses on
different basic block categories.
We show, for instance, that all the models we evaluated
have average errors higher than 30\% while predicting
vectorized basic blocks.

\end{abstract}