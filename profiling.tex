\section{Basic Block Profiling}

\subsection{Measurement Guarantees}
We now describe our approach to profiling basic block thoughput.
We use IACA's definition of throughput, i.e. 
the average number of cycles required to execute a basic block inside
an infinite loop.
We obtain the throughput for a basic block by measuring the unrolled latency,
from which we derive throughput; latency is measured by reading core cycle counter,
which is supported in recent versions x86-64 processors.

We make the following guarantees for our measurements.
\begin{enumerate*}
    \item All memory data reside in L1 cache.
    \item Measurements should be free of interrupts, context switches,
    and hyper-threading.
\end{enumerate*}
Admittedly in practice the memory accesses of many basic blocks do not hit L1 cache.
We nonetheless make sure that during our measurement all memory accesses hit L1 cache 
because it is the same assumption made by existing models;
and because of consistency: the throughput of these basic blocks (i.e. those with memory accesses)
is inherently dependent on the program execution context;
without knowing the precise context, we favor consistency over generality.

\textbf{Ensuring Measurement Invariants}.
We monitor the following statistics, in addition to core cycle counts,
during measurements.
\begin{enumerate}
    \item L1 Data Cache Read Misses
    \item L1 Data Cache Write Misses
    \item L1 Instruction Cache Misses
    \item Context Switches
\end{enumerate}
If any of the aforementioned counters are non-zero,
the measurement is discarded.
Each \textit{unrolled} basic block is timed 16 times by default,
and we accept a basic block for further analysis only if the basic block has at least 
8 \textit{clean}, identical timing.
All basic blocks are measured in machines with hyper-threading disabled.
We count the elapsed cycles by reading the core cycle counter.
Unlike reference cycle or time-stamp counter, core cycle counter is invariant
to performance scaling.

\subsection{Handling memory accesses}
Consider the basic block in figure \ref{fig:mem-ex},
which is used to compute the CRC code of input bytes in Gzip.
Highlighted instructions shows the flow of pointer values;
essentially \textit{bits} of \verb|rdx| are used to index into a lookup table, 
and the content of which is then used in the next iteration to 
update \verb|rdx|.
How would one profile the throughput of this basic block?
Without the original application context,
this basic block \textit{cannot} be directly executed.
Developer of Gzip familiar with this piece of code could 
properly setup a micro-benchmark by reproducing the
relevant lookup table used by this code snippet,
but such approach would not scale to the hundred of 
thousands of basic blocks in our data set, most of which contain memory accesses.
We needed some way to automatically generate the benchmarking environment
without apriori knowledge/assumption of the code being profiled.

\begin{figure}[h]
\begin{Verbatim}[commandchars=\#\{\}]
    add rdi, 1
    #hl{mov eax, edx}
    shr rdx, 8
    #hl{xor al, [rdi - 1]}
    movzx eax, al
    #hl{xor rdx, [8*rax + 0x4110a]}
    cmp rdi, rcx
\end{Verbatim}
\caption{Inner loop body of updcrc from Gzip.
This basic block cannot be directly executed because
of its memory accesses.}
\label{fig:mem-ex}
\end{figure}

\begin{figure}
\begin{algorithmic}

\Function{measure}{}
    \State $page \gets \Call{getRegisterValue}{rax}$
    \State $\Call{mmapToChosenPhysPage}{page, ...}$
    \State $\Call{initialize}{}$
    \State $\Call{executeUnrolledBasicBlock}{}$
    
    \Comment{After this point all memory accesses made by the basic block is legal
    and goes to L1 cache}
    
    \State $begin \gets \Call{readPerformanceCounters}{}$
    \State $\Call{initialize}{}$
    \State $\Call{executeUnrolledBasicBlock}{}$
    \State $end \gets \Call{readPerformanceCounters}{}$
    \State $\Call{analyzeAndReportCounters}{begin, end}$
\EndFunction

\Function{monitor}{}
    \State $pid \gets \Call{launchAndMonitor}{measure}$
    \State $numFaults \gets 0$
    \While{$true$}
        \State $memAddr \gets \Call{interceptSegFault}{pid}$
        \If{$\Call{isValidAddr}{memAddr}$}
            \State $page \gets \Call{getPageAddr}{memAddr}$
            \State $\Call{setRAX}{pid, page}$
            \State $\Call{setInstructionPointer}{pid, measure}$
            \State $\Call{resumeProcessExecution}{pid, measure}$
            \State $numFaults \gets numFaults + 1$
        \EndIf
        \If{$numFaults > maxNumFaults$}
            \State $\Call{kill}{pid}$
            \State \textbf{break}
        \EndIf
    \EndWhile
\EndFunction

\end{algorithmic}

\caption{Pseudocode of the our profiling routines. 
\textit{measure} runs in a child process monitored by
\textit{monitor}.
We execute an input basic block twice.
The first execution is used to discover virtual pages accessed
by the basic block, which are the then mapped to a single physical page.
We collect performance statistics in the second execution.}
\label{fig:code}
\end{figure}

%TODO: point out concretely why previous approach, including prefixing
% addresses with the array does not work
A tempting solution~\cite{ithemal} is rewriting
the basic block so that all memory accesses are
relative to a pre-allocated array.
For instance, \verb|mov [rdx], 1| is rewritten to
\verb|mov [array_address+rdx], 1|\footnote{
A cleverer solution is to
base the address off the \textit{midpoint} of the array
in anticipation of negative indices.
}. 
This approach however is not likely to produce a satisfying measurement
for our example.
In our example,
because of the hardcoded offset -- \verb|0x4110a|,
the rewriting scheme can reliably prevent crashes only when the pre-allocated array
is bigger than 260 kilobytes, much bigger than the L1-caches of existing microarchitectures.
Additionally, this approach has several issues:
\begin{itemize}
    \item Executing some basic blocks back to back can result in a 
    diverging (e.g. strided) access pattern. 
    One would need to pre-allocate an unrealistically large array
    in order to  accommodate for this pattern, 
    Even when a sufficiently large buffer could be allocated,
    this access pattern can result in cache misses, 
    which we want to avoid.
    \item The scheme does not take stack accesses into account; i.e. \verb|push|s and \verb|pop|s
    are not considered.
    \item Rewritten instructions can
    have larger encoding, which can affect throughput
    of front-end bound basic blocks.
\end{itemize}

We avoid these pitfalls with a new technique
that divides the profiling process into two stages:
\begin{enumerate*}
\item We profile the virtual pages accessed by
the basic block (were it to run without crashing)
and map those pages to a single physical page.
\item After the appropriate mappings are created,
the unrolled basic block is then executed normally under
the new page mapping.
\end{enumerate*}

\textbf{Remapping pages to legalize memory accesses}
In addition to legalizing all memory accesses made by the basic block,
the mapping stage also ensures that all memory accesses stay in
the L1 cache.
Figure~\ref{fig:code} shows the algorithm we use to do page mapping.
Mapping is done by executing the unrolled basic block in a forked process
 monitored by a parent process using \verb|ptrace|.
Each attempt to access an unmapped virtual page is intercepted by
the monitoring process, which then instructs the 
executing process to create the appropriate mapping
and to restart execution from the \textit{beginning}
with all registers, memory values,
as well as flag registers reinitialized.
Re-initialization is done to ensure that in the final measurement
stage the trace of memory addresses computed by the 
unrolled basic block is \textit{identical} to that from the mapping stage.
An alternative we considered was using a single process for both execution
and monitoring, but we ultimately disconsiderred this alternative due to
its complexity.

\subsection{Additional measurement optimizations}
We make additional optimizations to further increase our chance
of successfully obtaining clean measurements.

\textbf{Memory Initialization} 
Prior to the mapping-run,
we unmap all pages (except those containing the basic block).
This allows us to run basic blocks that inadvertently
writes to library code such as libc.
The physical page is
always initialized to be filled with a “moderately sized” constant
(we used \verb|0x12345600| in our experiments)
to accommodate for indirect memory accesses.
To see why this is necessary,
consider a basic block that first loads a pointer $p$ from memory
and then de-references $p$.
If the value of $p$ is too low (e.g. 0)
or too high
(i.e. bigger than the maximum memory that
a user space program is allowed to addressed),
then we will not be able to map the virtual page pointed by $p$.

\textbf{Handling Subnormal Numbers}
General speaking -- aside from division
and specialized instructions such as \verb|sqrt|
-- floating point instructions have constant timing
except when input of the operations are subnormal numbers.
In our experiments where we encountered subnormal numbers,
floating point operations can be slowed down by up to 20x.
To ``normalize'' our timing, we configured \verb|MXCSR| register
to disable gradual underflow.

\subsection{Deriving throughput from measurement}.
The basic strategy one usually takes to measure basic block throughput
is unrolling a basic block multiple times and divide the latency of the
unrolled basic block by the unroll factor.
Compared to running the basic block inside a loop,
unrolling has an advantage in that the measurement is not polluted by
the overhead incurred by looping.
A typical unroll factor is 100\cite{ithemal,uops}.
Such a high unroll factor can however incur significant
L1-icache misses for large basic blocks
(e.g. already unrolled inner-loop body of a GEMM kernel).
The issue with unrolling naively is that,
an unroll-factor small enough to fit the 
unrolled basic block into instruction cache is on the other hand
insufficient to hide the latency of the first few iterations.
We thus adopted a different strategy to cope with large basic blocks.

We unroll each basic block with two different unroll factors.
These factors should be large enough 
to warm-up the processor into steady state.
We then measure the latency of the two unrolled basic blocks,
calculate the difference in latency, and divide it 
by difference of the unroll factors.
The resulting number is the throughput of the basic block.
Consider a processor that enters steady state
after executing a basic block 10 iterations.
We measure the latency of this basic block expanded with
two unroll factors -- 10 and 20.
Suppose the resulting latency is 120 and 200 respectively,
we can then determine that the throughput of this basic block
is $\frac{200-120}{20-10} = 8$ cycles per iteration.


\begin{table}
\begin{tabular}{
|p{0.2\columnwidth}|p{0.18\columnwidth}|p{0.18\columnwidth}|p{0.18\columnwidth}|}
\hline (Additional) Optimizations &
Measured Throughput &
L1 D-Cache Misses &
L1 I-Cache Misses \\

\hline
None & Crashed & N/A & N/A \\

\hline
Page mapping & 6377.0 & 956 & 0 \\

\hline
Single physical page & 2273.7 & 0 & 0 \\

\hline
Disabling gradual underflow & 65.0 & 0 & 35 \\

\hline
Using smaller unroll factor & 59.0 & 0 & 0\\

\hline
\end{tabular}
\\
\caption{Measured throughput for a sample basic block when
different measurement optimizations are incrementally applied.
The basic block is extracted from one of the critical 
inner loop body of TensorFlow\cite{tensorflow}'s CNN training benchmark.}
\label{tab:ablation}
\end{table}