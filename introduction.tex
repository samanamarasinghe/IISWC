\section{Introduction}
%Many optimizations, whether manual or automatic,
%benefit from performance models.

Hardware performance models help programmers or compilers to 
produce high performing code.
For example, static machine code analyzers such as IACA~\cite{iaca} helps programmers
to estimate the performance of short code sequences
isolated from their source applications.
Production compilers such as LLVM~\cite{llvm} use cost models to guide various optimizations such as instruction scheduling, loop
vectorization and SLP vectorization~\cite{slp} (to determine whether a particular vectorization decision is beneficial).
%compiler optimizations; such as instruction scheduler and SLP vectorizer.
%instruction scheduler usto estimate the basic block throughput of different schedules and SLP vectorizer~\cite{slp} to determine whether a particular vectorization strategy is beneficial.
%and they also alleviate developers the burden of 
%properly setting up micro-benchmarking environment.
%Many compiler optimizations
%use cost models of one form or another:
%a typical automatic vectorizer, for instance,
%uses one to estimate the cost of vector packing
%in the presence of irregular memory accesses~\cite{goslp};
%instruction schedulers use one to estimate the latency
%and throughput of instructions~\cite{llvm-sched,gcc-sched}.

Development and validation of cost models are often overlooked compared to actual optimizations,
despite the fact that
the accuracy of the former ensures the effectiveness of the latter.
For instance, Mendis et al.\cite{goslp}
have shown an optimal\footnote{
With certain caveat.
In particular their formulation is optimal
only when the target machine only has vector-width of 2}
solution to SLP Vectorization\cite{slp}.
Parameterized by LLVM's vectorization cost model\cite{llvm-cost},
their algorithm is optimal insofar as the cost model is able to accurately 
capture the target architectures.
In some instances, their solutions incurred slowdowns due to inaccuracies of 
the cost models.
Hence, to make sure that compiler optimizations and performance engineers are properly guided,
a well tuned cost model is of importance.

However, modern cost models are not developed systematically and are not properly validated.
For example, parameters of these cost models 
are sometimes chosen haphazardly;
consider the following quote taken from a commit message from LLVM\cite{llvm}
regarding its x86-64 cost model\footnote{
\url{https://reviews.llvm.org/D46276}
}:
\begin{quote}
x86's fix to avoid SDIV/UDIV vectorization is clumsy
(multiplying the vector costs x20)
and appears to have been implemented before the cost of
scalarization was being realistically accounted for...
\end{quote}
This suggests that
developers cannot systematically improve these cost models
without a comprehensive benchmark to pinpoint the weakness of their models,
and have to resort to ad hoc bug fixes.

In this paper, we present a dataset
of basic blocks with relevant throughput measurements needed to validate x86-64 cost models used for
basic block throughput estimation. 
We also describe the throughput profiler we designed to collect these measurements,
with some guarantees that we will describe later.
While there have been multiple efforts~\cite{agner,uops,exegesis} to measure per-instruction
statistics for x86-64,
there has been no such dataset or measurement framework to validate basic block throughput models used by
compilers and machine code analyzers.
Basic block throughput cannot be computed accurately even when given perfect modelling of individual instructions,
due to opaque microarchitectural optimizations such as out-of-order execution and micro-op fusion.
One of the challenges of curating and profiling a set of basic blocks diverse enough to
uncover the intricacies of modern x86-64 processors is automatically profiling arbitrary
basic blocks.
This is much more than simply measuring the latency of an unrolled basic block,
since code snippets with memory accesses (which is the case for most basic blocks)
generally cannot be executed safely
outside the context of their source applications.
We used several techniques to profile these basic blocks out-of-context,
while guaranteeing certain variants that we believe are essential are accurate measurements (see Section~\ref{sec:invariants}).
Our ablation study (see Table~\ref{tab:full-ablation}) shows that, without these techniques,
a substantial amount of these basic blocks either simply cannot be profiled due to crashing
or produce misleading data.

%We describe a benchmark for validating performance models
%of x86-64 basic blocks.
%In this work we focus on developing a benchmark suite to throughput prediction.
In this paper, we make the following contributions:
\begin{itemize}
\item We propose a benchmark suite of over 300,000 basic blocks collected from a wide range of domains from numerical computation (e.g., OpenBLAS) to databases (SQLite).

\item We describe a fully automatic framework
for measuring the throughput of arbitrary basic blocks extracted 
from raw binaries without any user annotation or intervention.
We have used this framework to profile more than 2 million basic blocks.

\item We automatically classified the basic blocks
by their usage of different execution ports.
This classification allows developers to pinpoint weaknesses
of their performance models specific to certain patterns of x86-64 instructions.

\item We evaluated four existing cost models:
IACA~\cite{iaca}, llvm-mca~\cite{llvm-mca} (which exposes LLVM’s cost model used for instruction scheduling),
OSACA\cite{osaca}, and Ithemal\cite{ithemal}.
We break the strength and weakness of these cost models down on
different categories of basic blocks;
e.g. we show that all such tools
have average error higher than 30\% 
when used to analyze numerical kernels running on Haswell machines.

\end{itemize}